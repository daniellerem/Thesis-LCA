% Options for packages loaded elsewhere
\PassOptionsToPackage{unicode}{hyperref}
\PassOptionsToPackage{hyphens}{url}
%
\documentclass[
]{article}
\title{Opzet LCA MILC TREE}
\author{Daniëlle Remmerswaal}
\date{29-10-2021}

\usepackage{amsmath,amssymb}
\usepackage{lmodern}
\usepackage{iftex}
\ifPDFTeX
  \usepackage[T1]{fontenc}
  \usepackage[utf8]{inputenc}
  \usepackage{textcomp} % provide euro and other symbols
\else % if luatex or xetex
  \usepackage{unicode-math}
  \defaultfontfeatures{Scale=MatchLowercase}
  \defaultfontfeatures[\rmfamily]{Ligatures=TeX,Scale=1}
\fi
% Use upquote if available, for straight quotes in verbatim environments
\IfFileExists{upquote.sty}{\usepackage{upquote}}{}
\IfFileExists{microtype.sty}{% use microtype if available
  \usepackage[]{microtype}
  \UseMicrotypeSet[protrusion]{basicmath} % disable protrusion for tt fonts
}{}
\makeatletter
\@ifundefined{KOMAClassName}{% if non-KOMA class
  \IfFileExists{parskip.sty}{%
    \usepackage{parskip}
  }{% else
    \setlength{\parindent}{0pt}
    \setlength{\parskip}{6pt plus 2pt minus 1pt}}
}{% if KOMA class
  \KOMAoptions{parskip=half}}
\makeatother
\usepackage{xcolor}
\IfFileExists{xurl.sty}{\usepackage{xurl}}{} % add URL line breaks if available
\IfFileExists{bookmark.sty}{\usepackage{bookmark}}{\usepackage{hyperref}}
\hypersetup{
  pdftitle={Opzet LCA MILC TREE},
  pdfauthor={Daniëlle Remmerswaal},
  hidelinks,
  pdfcreator={LaTeX via pandoc}}
\urlstyle{same} % disable monospaced font for URLs
\usepackage[margin=1in]{geometry}
\usepackage{color}
\usepackage{fancyvrb}
\newcommand{\VerbBar}{|}
\newcommand{\VERB}{\Verb[commandchars=\\\{\}]}
\DefineVerbatimEnvironment{Highlighting}{Verbatim}{commandchars=\\\{\}}
% Add ',fontsize=\small' for more characters per line
\usepackage{framed}
\definecolor{shadecolor}{RGB}{248,248,248}
\newenvironment{Shaded}{\begin{snugshade}}{\end{snugshade}}
\newcommand{\AlertTok}[1]{\textcolor[rgb]{0.94,0.16,0.16}{#1}}
\newcommand{\AnnotationTok}[1]{\textcolor[rgb]{0.56,0.35,0.01}{\textbf{\textit{#1}}}}
\newcommand{\AttributeTok}[1]{\textcolor[rgb]{0.77,0.63,0.00}{#1}}
\newcommand{\BaseNTok}[1]{\textcolor[rgb]{0.00,0.00,0.81}{#1}}
\newcommand{\BuiltInTok}[1]{#1}
\newcommand{\CharTok}[1]{\textcolor[rgb]{0.31,0.60,0.02}{#1}}
\newcommand{\CommentTok}[1]{\textcolor[rgb]{0.56,0.35,0.01}{\textit{#1}}}
\newcommand{\CommentVarTok}[1]{\textcolor[rgb]{0.56,0.35,0.01}{\textbf{\textit{#1}}}}
\newcommand{\ConstantTok}[1]{\textcolor[rgb]{0.00,0.00,0.00}{#1}}
\newcommand{\ControlFlowTok}[1]{\textcolor[rgb]{0.13,0.29,0.53}{\textbf{#1}}}
\newcommand{\DataTypeTok}[1]{\textcolor[rgb]{0.13,0.29,0.53}{#1}}
\newcommand{\DecValTok}[1]{\textcolor[rgb]{0.00,0.00,0.81}{#1}}
\newcommand{\DocumentationTok}[1]{\textcolor[rgb]{0.56,0.35,0.01}{\textbf{\textit{#1}}}}
\newcommand{\ErrorTok}[1]{\textcolor[rgb]{0.64,0.00,0.00}{\textbf{#1}}}
\newcommand{\ExtensionTok}[1]{#1}
\newcommand{\FloatTok}[1]{\textcolor[rgb]{0.00,0.00,0.81}{#1}}
\newcommand{\FunctionTok}[1]{\textcolor[rgb]{0.00,0.00,0.00}{#1}}
\newcommand{\ImportTok}[1]{#1}
\newcommand{\InformationTok}[1]{\textcolor[rgb]{0.56,0.35,0.01}{\textbf{\textit{#1}}}}
\newcommand{\KeywordTok}[1]{\textcolor[rgb]{0.13,0.29,0.53}{\textbf{#1}}}
\newcommand{\NormalTok}[1]{#1}
\newcommand{\OperatorTok}[1]{\textcolor[rgb]{0.81,0.36,0.00}{\textbf{#1}}}
\newcommand{\OtherTok}[1]{\textcolor[rgb]{0.56,0.35,0.01}{#1}}
\newcommand{\PreprocessorTok}[1]{\textcolor[rgb]{0.56,0.35,0.01}{\textit{#1}}}
\newcommand{\RegionMarkerTok}[1]{#1}
\newcommand{\SpecialCharTok}[1]{\textcolor[rgb]{0.00,0.00,0.00}{#1}}
\newcommand{\SpecialStringTok}[1]{\textcolor[rgb]{0.31,0.60,0.02}{#1}}
\newcommand{\StringTok}[1]{\textcolor[rgb]{0.31,0.60,0.02}{#1}}
\newcommand{\VariableTok}[1]{\textcolor[rgb]{0.00,0.00,0.00}{#1}}
\newcommand{\VerbatimStringTok}[1]{\textcolor[rgb]{0.31,0.60,0.02}{#1}}
\newcommand{\WarningTok}[1]{\textcolor[rgb]{0.56,0.35,0.01}{\textbf{\textit{#1}}}}
\usepackage{graphicx}
\makeatletter
\def\maxwidth{\ifdim\Gin@nat@width>\linewidth\linewidth\else\Gin@nat@width\fi}
\def\maxheight{\ifdim\Gin@nat@height>\textheight\textheight\else\Gin@nat@height\fi}
\makeatother
% Scale images if necessary, so that they will not overflow the page
% margins by default, and it is still possible to overwrite the defaults
% using explicit options in \includegraphics[width, height, ...]{}
\setkeys{Gin}{width=\maxwidth,height=\maxheight,keepaspectratio}
% Set default figure placement to htbp
\makeatletter
\def\fps@figure{htbp}
\makeatother
\setlength{\emergencystretch}{3em} % prevent overfull lines
\providecommand{\tightlist}{%
  \setlength{\itemsep}{0pt}\setlength{\parskip}{0pt}}
\setcounter{secnumdepth}{-\maxdimen} % remove section numbering
\ifLuaTeX
  \usepackage{selnolig}  % disable illegal ligatures
\fi

\begin{document}
\maketitle

Structuur 1. Intro 2. Literatuur o Total error framework o Verschil MTMM
(2 fouten aan meetkant) ik: 1 fout aan meetkant, en 1 fout aan
representatiekant o 3. Methoden o MILC o Tree structuur o Simulatie van
data Populatiemodel Stap voor stap met code 4. Resultaten

\hypertarget{mentor-feedback-12-november}{%
\section{Mentor feedback (12
november)}\label{mentor-feedback-12-november}}

Send a version of your report to all students in your mentor group (Cc
the mentor) no later than 5 PM on the Friday before the meeting
(November 12th). Meeting is wednesday 17 november This whole meeting is
about providing and receiving feedback on some aspects of the research
report. The focus of the peer feedback is the reproducibility of the
research design. Based on the research proposal and feedback from the
BoS or your supervisor, you should be able to update your plan for your
thesis. It is not needed to have a full report ready by now. The
feedback you will give to your peer should be focused on the
reproducibility of the research design. That is, will you be able to
reproduce the results (which are probably not there yet) with the
information provided in the analytic strategy. It is unnecessary to give
feedback on the introduction, although it helps to have n obvious
research question and hypotheses.

\hypertarget{introduction}{%
\section{1. Introduction}\label{introduction}}

\hypertarget{problem}{%
\subsection{1.1. Problem}\label{problem}}

In many cases when official statistics are produced, only a subset from
a population is selected. An example is a statistic on the energy
consumption per economic sector. From a register containing all
buildings in the Netherlands, only establishments are selected, and
dwellings discarded. Information on energy consumption per building and
on the type of building and, if applicable, the economic sector, is
recorded in multiple data sources. To produce the statistic, the data
sources, which are not error-free, are combined. In a composite dataset,
a combination of registers and surveys on the same statistic, errors
become noticeable. In this type of situation we can distinguish between
selection errors (regarding the type of building) and classification
errors (of the economic sectors). Possible causes of errors are typos
during data entry, mismatching definitions (between the data collectors
and the CBS researchers), or not having up-to-date answers (Groen,
2012). It is important to estimate and correct errors as to identify and
minimize bias in statistics.

\hypertarget{gap-from-proposal}{%
\subsection{1.3. Gap (from proposal)}\label{gap-from-proposal}}

For the estimation of categorical errors of one particular type, or in
combination with `method error' (MTMM situation), multiple methods are
available to researchers. However, on approaches for situations where
misclassifications can be identified as being caused by different error
types (e.g.~selection or classification error), not much research has
been conducted. An unexplored potential solution for this situation, is
the use of a latent class tree in combination with the MILC method. In
this thesis this new latent class tree approach to the MILC method will
be compared to the ``normal'' MILC method, in which an extra class is
added for the wrongly selected units. The structure crucial to the two
methods is visualised in section 2.2. with the described energy
consumption per economic sector example.\\
\#\# 1.4. Research Question, hypotheses and expectations (from proposal)
This project aims to investigate whether the combination of the
MILC-method with a latent class tree approach is an appropriate
alternative to the MILC-method, to measure and correct for a combination
of selection and classification errors. How well those approaches
perform to estimate and correct the errors, will be investigated with a
simulation study. Expectation is that the latent class tree structure
gives more accurate estimates and more insight in the structure of data
with those two types of errors, but is less straightforward to apply.

To perform the simulations and analyse the results, the freely-available
opensource statistical software R will be used. In particular, the
package poLCA will be used for the data simulation and the latent class
analysis. (kopie zin)

\hypertarget{literatuur}{%
\section{2. Literatuur}\label{literatuur}}

\hypertarget{short-from-proposal}{%
\subsection{2.1. short (from proposal)}\label{short-from-proposal}}

Several methods have been developed to estimate and correct for errors
in categorical data. To estimate errors in data at one time-point, an
important method is the latent variable model approach by Biemer (2011),
with Latent Class Analysis (LCA) being specifically for errors in
categorical variables. For categorical data measured multiple times,
Hidden-Markov Models (HMM) can be used (Pavlopoulos \& Vermunt, 2015;
Pankowska, 2020). In survey research, systematic (caused by survey
design) and random measurement error can be simultaneously studied with
the multitrait-multimethod (MTMM) approach (Cernat and Oberski, 2018).
The latent variable model is applicable on composite datasets under the
assumption that the sources are locally independent. Those measurements
are then used as the `indicators' of a latent, unobserved, `true'
variable. The Multiple Imputing Latent Class (MILC) method uses a latent
class model to can also be used to correct the true value for
categorical errors, and sets the amount of latent classes equal to the
number of categories in the variable of interest (Boeschoten, 2019).

\hypertarget{lca}{%
\subsection{LCA}\label{lca}}

Latent Class models are finite mixture models (fmm). (Latent class
analysis is binomial fmm, latent profile analysis is gaussian fmm).

\begin{itemize}
\tightlist
\item
  Assumptions Two key model assumptions (X is the latent class variable)
\end{itemize}

\begin{enumerate}
\def\labelenumi{\arabic{enumi}.}
\tightlist
\item
  (MIXTURE ASSUMPTION) Joint distribution mixture of 2 class-specific
  distributions: P(scorepattern)=P(x=1)P(scorepattern\textbar X=1)+
  P(x=2)P(scorepattern\textbar X=2). Mixture of C classes.
\item
  (LOCAL INDEPENDENCE ASSUMPTION) Within class X=x, responses are
  independent. P(scorepattern\textbar X=1)=P(Q1=A1\textbar X=1) *
  P(Q2=A2\textbar X=1)* P(Q3=A3\textbar X=1). K manifest variables. Base
  calculations of posterior probabilities on mixture and local
  independence assumption.
\end{enumerate}

Classification via modal assignment: assign unit to latent class with
highest P(X=x\textbar Y=y)=P(Class\textbar scorepattern)

\begin{itemize}
\tightlist
\item
  History ``A general framework for categorical data analysis with
  discrete latent variables was proposed by Hagenaars (1990) and
  extended by Vermunt (1997).''
\end{itemize}

\hypertarget{total-error-framework}{%
\subsection{Total error framework}\label{total-error-framework}}

Total Survey Error is ``any error arising from the survey process that
contributed to the deviation of an estimate from its true parameter
value'' (Biemer, 2016). TSE all errors that can occur during design,
collection, processing, and analysis of survey data. We can divide TSE
between sampling and non-sampling errors. Sampling errors are due to
sampling a sample instead of a whole population. non-sampling errors
include measurements errors (specification(of concept in question),
sampling frame, nonresponse, measurement, data processing, and
modelling/estimation errors).

Especially compared to survey data, there is not much statistical
theories for assessing the uncertainty of register-based statistics(Holt
2007; in Zhang 2012). Zhang (2012) developed a Total Survey Error
framework for administrative datasources.

\textbf{Verschil MTMM (2 fouten aan meetkant) ik: 1 fout aan meetkant,
en 1 fout aan representatiekant}

\hypertarget{methoden}{%
\section{3. Methoden}\label{methoden}}

\hypertarget{analytical-strategy-overview-from-proposal}{%
\subsection{3.1. Analytical strategy Overview (from
proposal)}\label{analytical-strategy-overview-from-proposal}}

. Simulated datasets (N=10.000) . Variables: measurements of different
datasources on same attribute . Classes: the categories in variable of
interest . Variation: in the amount of selection and classification
errors, and their relative proportions To be able to evaluate the
performance of the beforementioned extensions of the MILC method to
estimate and correct errors, simulated data will be used since herewith
`true values' are available. To perform the simulations and analyse the
results, the freely-available opensource statistical software R will be
used. In particular, the package poLCA will be used for the data
simulation and the latent class analysis.

The simulated composite datasets will contain both selection and
classification errors (of varying levels). In the simulated dataset the
latent classes are the groups that are potentially misselected or
misclassified, and each variable contains information on the categories
from one data source. Next, the MILC method and the MILC-method with a
latent class tree approach are both applied on the simulated datasets.
The performance of the classifications of both methods is evaluated by
comparing it to the original simulated datasets. The accuracy of the
methods will be compared in terms of bias and confidence validity of the
statistic under investigation, the proportion of the classes, and on how
well they estimate the (relative) size of the two errors.

the latent class tree approach allows for the selection errors (building
type) and classification errors (economic sectors) to be distinguished.
Without this tree structure, the misclassified buildings will simply get
allocated to an additional class.

\hypertarget{milc}{%
\subsection{MILC}\label{milc}}

How many imputations? see recommendations Boeschoten (2017).

\hypertarget{milc-with-tree-step}{%
\subsection{MILC with tree step}\label{milc-with-tree-step}}

\hypertarget{latent-class-trees-mattis-van-den-bergh}{%
\subsubsection{Latent class trees (Mattis van den
bergh)}\label{latent-class-trees-mattis-van-den-bergh}}

The data is split and structured into classes by method of a sequential
comparison of 1- and 2-class models. The top-down approach continues
until the information criterion (e.g.~BIC) no longer chooses 2-class
models over 1-class models. The splits are performed based on the
posterior class membership probabilities of each new class (child nodes)
conditional on the class before the splitting (parent node). For the BIC
(-2logL+log{[}N{]}P), N is the total sample size, not the sample size of
the concerned node, and P the number of parameters. A larger BIC value
indicates a more important split (a larger improvement of fit). To
prevent label switching (order depends on random starting values), the
larger class is given number 1 (in the picture the left branch) and the
smaller class number 2 (right branch). classification performance in
terms of R2Entropy, as a measure. Advantages: clear insight in splitting
of the classes and how models with different number of classes are
related to each other, ``the model selection problem reduces to deciding
whether a particular split should be accepted'' or not. ` The latent
class tree approach is an alternative and insightful way to perform a
latent class Bergh, M. van den, Kollenburg, G.H. van, \& Vermunt, J.K.
Deciding on the starting number of classes of a Latent Class Tree.
Non-binary split at the root node of the tree is useful. This approach
is then an intermediate between standard LC and LC tree analysis.

\hypertarget{polca-package}{%
\subsection{poLCA package}\label{polca-package}}

The poLCA package is an R software package for the estimation of latent
class (regression) models for polytomous outcome variables. It is
currently the only R package that allows the use of polytomous outcomes,
and covariates. Other R packages for categorical outcome variables can
only use binary outcomes.

Potential shortcomings: poLCA recodes NA's to a seperate category
(zero's) and does not make use of imputation methods.

\begin{itemize}
\tightlist
\item
  explain: how does it work? (avoid overlap with literature) LC models
  are typically estimated by maximum likelihood, \ldots{} log-likelihood
  function \ldots{} . EM algorithm, where EM stands for estimation and
  maximization.
\end{itemize}

``The latent class regression model further enables the researcher to
estimate the effects of covariates on predicting latent class
membership.''

``poLCA uses expectation-maximization and Newton-Raphson algorithms to
find maximum likelihood estimates of the model parameters.''

\begin{Shaded}
\begin{Highlighting}[]
\CommentTok{\#install.packages("poLCA") \#install if not done yet}
\FunctionTok{library}\NormalTok{(poLCA) }\CommentTok{\#load the package into memory}
\end{Highlighting}
\end{Shaded}

\begin{verbatim}
## Loading required package: scatterplot3d
\end{verbatim}

\begin{verbatim}
## Loading required package: MASS
\end{verbatim}

\begin{itemize}
\tightlist
\item
  label problem ``Because the latent classes are unordered categories,
  the numerical order of the estimated latent classes in the model
  output is arbitrary, and is determined solely by the start values of
  the EM algorithm. (Linzer and Lewis, 2011, poLCA, 4.6) try to fix with
  label.switching package? more succesful approach is with
  poLCA.reorder(polcamodel\$probs.start, c(1,3,2)) \#example to switch
  order of second and third class. needed are the start probabilities of
  a model. So this is only a post-hoc fix, and not a general solution.
\end{itemize}

Basics of poLCA simulation.

\begin{Shaded}
\begin{Highlighting}[]
\NormalTok{basic.sim }\OtherTok{\textless{}{-}} \FunctionTok{poLCA.simdata}\NormalTok{(}
  \AttributeTok{N=}\DecValTok{5000}\NormalTok{, }\CommentTok{\#standard is 5000}
  \AttributeTok{nclass =} \DecValTok{3}\NormalTok{, }\CommentTok{\#nr of latent classes, standard is 2}
  \AttributeTok{ndv =} \DecValTok{4}\NormalTok{, }\CommentTok{\#nr of manifest variables, standard is 2}
\NormalTok{)}
\NormalTok{basic.sim}\SpecialCharTok{$}\NormalTok{P}
\end{Highlighting}
\end{Shaded}

\begin{verbatim}
## [1] 0.05770074 0.37173654 0.57056271
\end{verbatim}

\begin{itemize}
\tightlist
\item
  proportions per class If not specified, the proportions per class are
  specified randomly (like in the example above). P can be specified
  with the proportions for each of the latent classes.
\end{itemize}

\begin{Shaded}
\begin{Highlighting}[]
\NormalTok{basic.sim2 }\OtherTok{\textless{}{-}} \FunctionTok{poLCA.simdata}\NormalTok{(}\AttributeTok{N=}\DecValTok{5000}\NormalTok{,}\AttributeTok{nclass=}\DecValTok{3}\NormalTok{,}\AttributeTok{ndv=}\DecValTok{4}\NormalTok{,}\AttributeTok{P=}\FunctionTok{c}\NormalTok{(}\FloatTok{0.4}\NormalTok{, }\FloatTok{0.3}\NormalTok{, }\FloatTok{0.3}\NormalTok{))}
\NormalTok{basic.sim2}\SpecialCharTok{$}\NormalTok{P}
\end{Highlighting}
\end{Shaded}

\begin{verbatim}
## [1] 0.4 0.3 0.3
\end{verbatim}

However, the specified proportions of the classes are `overruled' when
covariates are present.

\begin{Shaded}
\begin{Highlighting}[]
\CommentTok{\#two covariates added}
\FunctionTok{set.seed}\NormalTok{(}\DecValTok{123}\NormalTok{)}
\NormalTok{basic.sim3 }\OtherTok{\textless{}{-}} \FunctionTok{poLCA.simdata}\NormalTok{(}\AttributeTok{N=}\DecValTok{5000}\NormalTok{,}\AttributeTok{nclass=}\DecValTok{3}\NormalTok{,}\AttributeTok{ndv=}\DecValTok{4}\NormalTok{,}\AttributeTok{P=}\FunctionTok{c}\NormalTok{(}\FloatTok{0.4}\NormalTok{, }\FloatTok{0.3}\NormalTok{, }\FloatTok{0.3}\NormalTok{), }\AttributeTok{niv =} \DecValTok{2}\NormalTok{)}
\FunctionTok{head}\NormalTok{(basic.sim3}\SpecialCharTok{$}\NormalTok{dat) }\CommentTok{\#four manifest variables and two covariates}
\end{Highlighting}
\end{Shaded}

\begin{verbatim}
##   Y1 Y2 Y3 Y4         X1         X2
## 1  1  3  2  1  0.8377870  1.3424605
## 2  2  4  2  5  0.1533731  0.1235129
## 3  2  1  2  4 -1.1381369  0.8581478
## 4  1  4  2  5  1.2538149 -1.4054279
## 5  3  1  2  5  0.4264642 -0.0144571
## 6  2  4  3  1 -0.2950715  0.5455757
\end{verbatim}

\begin{Shaded}
\begin{Highlighting}[]
\NormalTok{basic.sim3}\SpecialCharTok{$}\NormalTok{P }\CommentTok{\#due to the covariates, the specified class proportions are overruled}
\end{Highlighting}
\end{Shaded}

\begin{verbatim}
## [1] 0.4999831 0.1680110 0.3320059
\end{verbatim}

\begin{Shaded}
\begin{Highlighting}[]
\NormalTok{basic.sim3}\SpecialCharTok{$}\NormalTok{b }\CommentTok{\#randomly drawn covariate structure with integers from {-}2 to 2}
\end{Highlighting}
\end{Shaded}

\begin{verbatim}
##      [,1] [,2]
## [1,]   -1   -1
## [2,]   -1   -2
## [3,]    0   -2
\end{verbatim}

The class-conditional outcome probabilities of the manifest variables
can be specified.

\begin{Shaded}
\begin{Highlighting}[]
\CommentTok{\#specify outcome probabilities of the manifest variables }
\FunctionTok{set.seed}\NormalTok{(}\DecValTok{123}\NormalTok{)}
\CommentTok{\#probabilities for three classes for four manifest variables}
\NormalTok{probs }\OtherTok{\textless{}{-}} \FunctionTok{list}\NormalTok{(}\FunctionTok{matrix}\NormalTok{(}\FunctionTok{c}\NormalTok{(}\FloatTok{0.9}\NormalTok{,}\FloatTok{0.05}\NormalTok{,}\FloatTok{0.05}\NormalTok{,}
                        \FloatTok{0.05}\NormalTok{,}\FloatTok{0.9}\NormalTok{,}\FloatTok{0.05}\NormalTok{,}
                        \FloatTok{0.05}\NormalTok{,}\FloatTok{0.05}\NormalTok{,}\FloatTok{0.9}\NormalTok{ ), }\AttributeTok{ncol=}\DecValTok{3}\NormalTok{,   }\AttributeTok{byrow=}\ConstantTok{TRUE}\NormalTok{), }\CommentTok{\# Y1}
               \FunctionTok{matrix}\NormalTok{(}\FunctionTok{c}\NormalTok{(}\FloatTok{0.9}\NormalTok{,}\FloatTok{0.05}\NormalTok{,}\FloatTok{0.05}\NormalTok{,}
                        \FloatTok{0.05}\NormalTok{,}\FloatTok{0.9}\NormalTok{,}\FloatTok{0.05}\NormalTok{,}
                        \FloatTok{0.05}\NormalTok{,}\FloatTok{0.05}\NormalTok{,}\FloatTok{0.9}\NormalTok{ ), }\AttributeTok{ncol=}\DecValTok{3}\NormalTok{, }\AttributeTok{byrow=}\ConstantTok{TRUE}\NormalTok{), }\CommentTok{\# Y2}
               \FunctionTok{matrix}\NormalTok{(}\FunctionTok{c}\NormalTok{(}\FloatTok{0.90}\NormalTok{,}\FloatTok{0.05}\NormalTok{,}\FloatTok{0.05}\NormalTok{,}
                        \FloatTok{0.05}\NormalTok{,}\FloatTok{0.9}\NormalTok{,}\FloatTok{0.05}\NormalTok{,}
                        \FloatTok{0.05}\NormalTok{,}\FloatTok{0.05}\NormalTok{,}\FloatTok{0.9}\NormalTok{ ), }\AttributeTok{ncol=}\DecValTok{3}\NormalTok{, }\AttributeTok{byrow=}\ConstantTok{TRUE}\NormalTok{), }\CommentTok{\#Y3}
               \FunctionTok{matrix}\NormalTok{(}\FunctionTok{c}\NormalTok{(}\FloatTok{0.05}\NormalTok{,}\FloatTok{0.05}\NormalTok{,}\FloatTok{0.90}\NormalTok{,}
                        \FloatTok{0.05}\NormalTok{,}\FloatTok{0.9}\NormalTok{,}\FloatTok{0.05}\NormalTok{,}
                        \FloatTok{0.90}\NormalTok{,}\FloatTok{0.05}\NormalTok{,}\FloatTok{0.05}\NormalTok{ ), }\AttributeTok{ncol=}\DecValTok{3}\NormalTok{, }\AttributeTok{byrow=}\ConstantTok{TRUE}\NormalTok{))}\CommentTok{\#Y4}
\NormalTok{basic.sim4 }\OtherTok{\textless{}{-}} \FunctionTok{poLCA.simdata}\NormalTok{(}\AttributeTok{N=}\DecValTok{5000}\NormalTok{,}\AttributeTok{nclass=}\DecValTok{2}\NormalTok{,}\AttributeTok{ndv=}\DecValTok{2}\NormalTok{,}\AttributeTok{P=}\FunctionTok{c}\NormalTok{(}\FloatTok{0.4}\NormalTok{, }\FloatTok{0.3}\NormalTok{, }\FloatTok{0.3}\NormalTok{), }\AttributeTok{probs=}\NormalTok{ probs)}
\NormalTok{basic.sim4}\SpecialCharTok{$}\NormalTok{probs}
\end{Highlighting}
\end{Shaded}

\begin{verbatim}
## [[1]]
##      [,1] [,2] [,3]
## [1,] 0.90 0.05 0.05
## [2,] 0.05 0.90 0.05
## [3,] 0.05 0.05 0.90
## 
## [[2]]
##      [,1] [,2] [,3]
## [1,] 0.90 0.05 0.05
## [2,] 0.05 0.90 0.05
## [3,] 0.05 0.05 0.90
## 
## [[3]]
##      [,1] [,2] [,3]
## [1,] 0.90 0.05 0.05
## [2,] 0.05 0.90 0.05
## [3,] 0.05 0.05 0.90
## 
## [[4]]
##      [,1] [,2] [,3]
## [1,] 0.05 0.05 0.90
## [2,] 0.05 0.90 0.05
## [3,] 0.90 0.05 0.05
\end{verbatim}

As can be seen the probs argument overrules the specification of the
number of classes and the number of manifest variables.

\hypertarget{data-simulation}{%
\subsection{Data simulation}\label{data-simulation}}

The data with the two types of errors, the beforementioned selection and
classification errors.

\begin{Shaded}
\begin{Highlighting}[]
\FunctionTok{library}\NormalTok{(poLCA) }
\FunctionTok{options}\NormalTok{(}\AttributeTok{scipen =} \DecValTok{999}\NormalTok{) }\CommentTok{\# remove scientific notation}
\FunctionTok{set.seed}\NormalTok{(}\DecValTok{123}\NormalTok{) }\CommentTok{\#set.seed for replicability}
\end{Highlighting}
\end{Shaded}

N = 5000 will be used. Boeschoten (2017) demonstrated that N=10.000 does
not yield different/better/\ldots{} estimates/results.

\hypertarget{application}{%
\section{4. Application}\label{application}}

\hypertarget{introduction-to-problem}{%
\subsection{4.1. Introduction to
problem}\label{introduction-to-problem}}

``Determining the correct activity code of economic units is often
rather difficult and prone to errors (e.g., Christensen 2008).'' Source:
Delden, Scholtus and Burger (2015 \& 2016)

\hypertarget{household-statistics}{%
\subsection{4.2. household statistics}\label{household-statistics}}

The household is an important statistical unit, but does not exist in
administrative registers, and thus has to be created by statistical
agencies. Errors are thus unavoidable. *general theory (maybe in
literature?) Registers themselves are generally not created and
maintained by statistical agencies themselves, but by external owners
(e.g.~governments or businesses) for administrative purposes. Zhang
(2009) developed a unit-error theory to provide ``a framework for
evaluating the statistical accuracy of these register-based household
statistics''. In general, there is a lack of theories to evaluate the
quality of register statistics, especially as compared to survey
methodology. Related is the flawed assumption that registers are
error-free, which they are not. While sampling errors are ofcourse
absent, there are other non-sampling errors: ``over- and under-coverage,
lack of relevance, misclassification, delays and mistakes in the data
registration process, inconsistency across the administrative sources,
and not the least missing data''. This is important

\end{document}
